\documentclass{article}
\usepackage[utf8]{inputenc}
\usepackage[T2A]{fontenc}
\usepackage{xcolor}
\usepackage[english,russian]
{babel}
\usepackage{amsfonts}
\usepackage{listings}

\definecolor{codegreen}{rgb}{0,0.6,0}
\definecolor{codegray}{rgb}{0.5,0.5,0.5}
\definecolor{codepurple}{rgb}{0.58,0,0.82}
\definecolor{backcolour}{rgb}{0.95,0.95,0.92}

\lstdefinestyle{mystyle}{
    backgroundcolor=\color{backcolour},   
    commentstyle=\color{codegreen},
    keywordstyle=\color{magenta},
    numberstyle=\tiny\color{codegray},
    stringstyle=\color{codepurple},
    basicstyle=\ttfamily\footnotesize,
    breakatwhitespace=false,         
    breaklines=true,
    captionpos=b,                    
    keepspaces=true,                 
    numbers=left,                    
    numbersep=5pt,                  
    showspaces=false,                
    showstringspaces=false,
    showtabs=false,                  
    tabsize=2
}
\lstset{style=mystyle, language=[Auto]Lisp, inputencoding=cp1251}

\begin{document}
\Large\textbf{Лабораторная работа №3}

\textbf{Выполнил: Кузнецов Павел М3207}
\vspace{5mm}

\textbf{Задача 1.2}
Предъявите доверительный интервал уровня $1 - \alpha$ для указанного параметра при данных предположениях (с обоснованиями). Сгенерируйте 2 выборки объёма объёма 25 и посчитайте доверительный интервал. Повторить 1000 раз. Посчитайте, сколько раз 95-процентный доверительный интервал покрывает реальное значение параметра. То же самое сделайте для объема выборки 10000. Как изменился результат? Как объяснить?

Задача представлена в 3 вариантах. Везде даны две независимые выборки X, Y из нормальных распределений $ N(\mu_1, \sigma_1^2), N(\mu_2, \sigma_2^2)$ объема n, m соответственно. Сначала указывается оцениваемая функция, потом данные об остальных параметрах, затем параметры эксперимента и подсказки.

$\tau = \mu_1 - \mu_2; \sigma_1^2=\sigma_2^2$ неизвестна; $\mu_1=2, \mu_2=1, \sigma_1^2=\sigma_2^2=1$; воспользуйтесь функцией $$\sqrt{\frac{mn(m+n-2)}{m+n}}\frac{\bar X - \bar Y - \tau}{\sqrt{nVar(X)+mVar(Y)}}$$
где Var(.) - выборочная смещенная дисперсия. Смотрите в сторону распределения Стьюдента.

\textbf{Решение:}
    Из функции ... приходим путём преобразований:
    $$\sqrt{\frac{mn(m+n-2)}{m+n}}\frac{\bar X - \bar Y - \tau}{\sqrt{nVar(X)+mVar(Y)}}
    \sim N(0,1)\sqrt{\frac{n+m-2}{nVar(X) + mVar(Y)}}$$
    Получившаяся функция имеет распределение Стьюдента с степенью свободы n+m-2

    Построим доверительный интервал уровня $1-\alpha$:
    $$P(\bar{X} - \bar{Y} - \frac{q\sqrt{(nVar(X)+mVar(Y))(m+n)}}{\sqrt{mn(m+n-2)}} \leq \tau $$
    $$\leq \overline{X} - \overline{Y} + \frac{q\sqrt{(nVar(X)+mVar(Y))(m+n)}}{\sqrt{mn(m+n-2)}}) = 1-\alpha$$
    где q - квантиль распределения Стьюдента для установленных степени свободы и уровня.

    Эксперимент:

    \begin{lstlisting}[language=Python, mathescape=true, breaklines=true]
import numpy
import scipy.stats
nu_1=2
nu_2=1
disp=1
n1 = 25
n2 = 10000
k = 1000
counter=0

q = scipy.stats.t.ppf(q = 0.975, df = 2 * n1 - 2)
precalculated_half = q * numpy.sqrt(2.0/(n1*(2*n1-2)))
for i in range (k):
    X_sample = numpy.random.normal(size=n1, loc = nu_1, scale=disp)
    Y_sample = numpy.random.normal(size=n1, loc= nu_2, scale=disp)
    lower_limit = numpy.mean(X_sample) - numpy.mean(Y_sample) - q * precalculated_half*numpy.sqrt(n1 * (numpy.var(X_sample, ddof=1) + numpy.var(Y_sample, ddof=1)))
    higher_limit = numpy.mean(X_sample) - numpy.mean(Y_sample) + q * precalculated_half*numpy.sqrt(n1 * (numpy.var(X_sample, ddof=1) + numpy.var(Y_sample, ddof=1)))
    if(lower_limit < (nu_1 - nu_2) and (nu_1 - nu_2) < higher_limit):
        counter+=1
print(counter)

q = scipy.stats.t.ppf(q = 0.975, df = 2 * n2 - 2)
precalculated_half = q * numpy.sqrt(2.0/(n2*(2*n2-2)))
for i in range (k):
    X_sample = numpy.random.normal(size=n2, loc = nu_1)
    Y_sample = numpy.random.normal(size=n2, loc= nu_2)
    lower_limit = numpy.mean(X_sample) - numpy.mean(Y_sample) - q * precalculated_half*numpy.sqrt(n2 * (numpy.var(X_sample) + numpy.var(Y_sample)))
    higher_limit = numpy.mean(X_sample) - numpy.mean(Y_sample) + q * precalculated_half*numpy.sqrt(n2 * (numpy.var(X_sample) + numpy.var(Y_sample)))
    if(lower_limit < (nu_1 - nu_2) and (nu_1 - nu_2) < higher_limit):
        counter+=1
print(counter)
    \end{lstlisting}
    Получившиеся результаты:

    Частота попадания реального значения в построенный доверительный интервал для выборки объема 25: 965

    
    Частота попадания реального значения в построенный доверительный интервал для выборки объема 10000: 973

    Частота немного выросла, что связывается с увеличением n, но в целом вероятность покрытия доверительного интервала остаётся близкой к постоянной, и слабо зависит от величины выборки.
    
\textbf{Задача 2.5}
Постройте асимптотический доверительный интервал уровня $ 1-\alpha$ для указанного параметра. Проведите эксперимент по схеме, аналогичной первой задаче.

Класс распределений: $Geom(p)$, оценить параметр p. Эксперимент для p = 0.7;
$$E(X)=\frac{1}{p} => \widehat{p}=\frac{1}{\bar{X}}$$
$$\sigma(\bar{X}) = \sqrt{\frac{p(1-p)}{n}}.$$

Тогда асимптотический доверительный интервал для параметра $p$ в $Geom(p)$ с уровнем доверия $1-\alpha$:

$$P(\widehat{p} - z_{\alpha/2} \cdot \sqrt{\frac{\widehat{p}(1-\widehat{p})}{n}} \leq p \leq\widehat{p} + z_{\alpha/2} \cdot \sqrt{\frac{\widehat{p}(1-\widehat{p})}{n}})$$ 

Эксперимент:

\begin{lstlisting}[language=Python, mathescape=true, breaklines=true]
import numpy
import scipy.stats
p = 0.7
n1 = 25
n2 = 10000
k = 1000
counter=0

q = scipy.stats.norm.ppf(q = 0.95)
for i in range (k):
    X_sample = scipy.stats.geom.rvs(p, size=n1)
    p_score = 1.0/numpy.mean(X_sample)
    lower_limit = p_score - q * numpy.sqrt(p_score * (1 - p_score)/n1)
    higher_limit = p_score + q * numpy.sqrt(p_score * (1 - p_score)/n1)
    if(lower_limit < p and p < higher_limit):
        counter+=1
print(counter)

for i in range (k):
    X_sample = scipy.stats.geom.rvs(p, size=n2)
    p_score = 1.0/numpy.mean(X_sample)
    lower_limit = p_score - q * numpy.sqrt(p_score * (1 - p_score)/n2)
    higher_limit = p_score + q * numpy.sqrt(p_score * (1 - p_score)/n2)
    if(lower_limit < p and p < higher_limit):
        counter+=1
print(counter)
    \end{lstlisting}
    Получившиеся результаты:
    Частота попадания реального значения в построенный доверительный интервал для выборки объема 25: 952

    
    Частота попадания реального значения в построенный доверительный интервал для выборки объема 10000: 979

    Вывод аналогичный.
    

\end{document}