\documentclass{article}
\usepackage[utf8]{inputenc}
\usepackage[T2A]{fontenc}
\usepackage{xcolor}
\usepackage[english,russian]
{babel}
\usepackage{amsfonts}
\usepackage{listings}

\definecolor{codegreen}{rgb}{0,0.6,0}
\definecolor{codegray}{rgb}{0.5,0.5,0.5}
\definecolor{codepurple}{rgb}{0.58,0,0.82}
\definecolor{backcolour}{rgb}{0.95,0.95,0.92}

\lstdefinestyle{mystyle}{
    backgroundcolor=\color{backcolour},   
    commentstyle=\color{codegreen},
    keywordstyle=\color{magenta},
    numberstyle=\tiny\color{codegray},
    stringstyle=\color{codepurple},
    basicstyle=\ttfamily\footnotesize,
    breakatwhitespace=false,         
    breaklines=true,
    captionpos=b,                    
    keepspaces=true,                 
    numbers=left,                    
    numbersep=5pt,                  
    showspaces=false,                
    showstringspaces=false,
    showtabs=false,                  
    tabsize=2
}
\lstset{style=mystyle, language=[Auto]Lisp, inputencoding=cp1251}

\begin{document}
\Large\textbf{Лабораторная работа №5}

\textbf{Выполнил: Кузнецов Павел М3207}

\textbf{Вариант 3}
\vspace{5mm}

\textbf{Условие:}

Вариант 3. В файле приведены данные о мужской обуви.

1. Постройте линейную модель, где в качестве независимых переменных выступают расход в городе, расход на шоссе, мощность (вместе со свободным коэффициентом), зависимой – цена.

2. Проверьте следующие подозрения:

a)Чем больше мощность, тем больше цена

b)Цена изменяется в зависимости от расхода в городе

c)Проверьте гипотезу $H_0$ о равенстве одновременно нулю коэффициентов при расходе в городе и расходе на шоссе против альтернативы $H1 = \bar H0$
\vspace{5mm}

\textbf{Решение:}

Вычисляем оценки коэффициентов линейной модели:
\begin{lstlisting}[language=Python, mathescape=true, breaklines=true]
import pandas as pd
import numpy
import scipy.stats as stats

data = pd.read_csv('/Users/frogge/proggs/mathstatLabs/lab-5/cars93.csv')


X = numpy.column_stack((numpy.ones(len(data)), data["MPG.city"], data["MPG.highway"], data["Horsepower"]))
Y = data['Price']

coefficients = numpy.linalg.inv(X.T.dot(X)).dot(X.T).dot(Y)  #matrix method
\end{lstlisting}   

Получившиеся коэффициенты:

overall: 6.68867892

city: -0.03860018

highway: -0.17885863

horsepower: 0.13131383

Остаточная дисперсия, доверительные интервалы для оценок коэффициентов:
\begin{lstlisting}[language=Python, mathescape=true, breaklines=true]
resvariance = numpy.sum((Y - X.dot(coefficients))**2) / (len(data) - len(coefficients))

covariance = resvariance * numpy.linalg.inv(X.T.dot(X))
err = numpy.sqrt(numpy.diag(covariance))
t = stats.t.ppf(0.975, len(data) - len(coefficients))
dov_intervals = numpy.column_stack((coefficients - t * err, coefficients + t * err))
\end{lstlisting} 

Получившиеся значения:

Остаточная дисперсия: 35.694685733186155

Доверительные интервалы для коэффициентов:

[[-5.21020505 18.58756289]

 [-0.74812245  0.67092209]
 
 [-0.88350169  0.52578444]
 
 [ 0.09931121  0.16331646]]

Вычисляем коэффициент детерминации:
\begin{lstlisting}[language=Python, mathescape=true, breaklines=true]
NSM = numpy.sum((Y - numpy.mean(Y))**2)
ESM = numpy.sum((X.dot(coefficients) - numpy.mean(Y))**2)
R2 = ESM / NSM
\end{lstlisting} 
Коэффициент детерминации: 0.6299138920082498

\textbf{2. Проверка гипотез:}
\begin{lstlisting}[language=Python, mathescape=true, breaklines=true]
uneededv, p = stats.ttest_ind(Y, data["Horsepower"])
if p < 0.05:
    print("first hypothesis incorrect")
else:
    print("first hypothesis correct")


uneededv, p = stats.ttest_ind(Y, data["MPG.city"])
if p < 0.05:
    print("second hypothesis incorrect")
else:
    print("second hypothesis correct")

uneededv, pc = stats.f_oneway(Y, data["MPG.city"])
uneededv, ph = stats.f_oneway(Y, data["MPG.highway"])
if pc < 0.05 or ph < 0.05:
    print("third hypothesis incorrect")
else:
    print("third hypothesis correct")
\end{lstlisting}
Вывод:

Первая гипотеза отвергается: мощность влияет на цену

Вторая гипотеза отвергается: расход в городе влияет на цену

Третья гипотеза отвергается: цена зависит от расхода в городе и/или от расхода на шоссе
\end{document}
