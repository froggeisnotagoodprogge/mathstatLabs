\documentclass{article}
\usepackage[utf8]{inputenc}
\usepackage[T2A]{fontenc}
\usepackage{xcolor}
\usepackage[english,russian]
{babel}
\usepackage{amsfonts}
\usepackage{listings}

\definecolor{codegreen}{rgb}{0,0.6,0}
\definecolor{codegray}{rgb}{0.5,0.5,0.5}
\definecolor{codepurple}{rgb}{0.58,0,0.82}
\definecolor{backcolour}{rgb}{0.95,0.95,0.92}

\lstdefinestyle{mystyle}{
    backgroundcolor=\color{backcolour},   
    commentstyle=\color{codegreen},
    keywordstyle=\color{magenta},
    numberstyle=\tiny\color{codegray},
    stringstyle=\color{codepurple},
    basicstyle=\ttfamily\footnotesize,
    breakatwhitespace=false,         
    breaklines=true,
    captionpos=b,                    
    keepspaces=true,                 
    numbers=left,                    
    numbersep=5pt,                  
    showspaces=false,                
    showstringspaces=false,
    showtabs=false,                  
    tabsize=2
}
\lstset{style=mystyle, language=[Auto]Lisp, inputencoding=cp1251}

\begin{document}
\Large\textbf{Лабораторная работа №4}

\textbf{Выполнил: Кузнецов Павел М3207}

\textbf{Вариант 2}
\vspace{5mm}

\textbf{Условие:}
В файле представлены данные о мобильных телефонах.

1. Разумно ли считать, что емкость аккумулятора распределена равномерно?

2. Верно ли, что телефонов с поддержкой 3G больше моделей с Wi-Fi? А разнится ли количество телефонов с touch screen от моделей с двумя сим-картами? На каждый вопрос по тесту.

3. Есть подозрение, что цена зависит от объема оперативной памяти. Проверите данное утверждение.
\vspace{5mm}

\textbf{Решение:}

Формализация задачи:

1. Разумно ли считать, что емкость аккумулятора имеет нормальное распределение, гипотеза $H_0 - $ емкость статистически распределена нормально.

\begin{lstlisting}[language=Python, mathescape=true, breaklines=true]
data = pd.read_csv('/Users/frogge/proggs/mathstatLabs/lab-4/mobile_phones.csv')

unkwn, p = stats.kstest(data['battery_power'], 'norm')
if p > 0.05:
    print('the h0 hypothesis is correct')
else:
    print('the h0 hypothesis is incorrect')

\end{lstlisting}   
Получившиеся результаты:

Гипотеза $H_0$ неверна.

Проверяем гипотезу с помощью критерия согласия Колмогорова

2. Верно ли, что телефонов с поддержкой 3G больше моделей с Wi-Fi, разнится ли количество телефонов с touch screen от моделей с двумя сим-картами:
\begin{lstlisting}[language=Python, mathescape=true, breaklines=true]
unneeded, p = stats.ttest_ind(data['three_g'], data['wifi'])
print(p)  # p from Student'test
if p > 0.05:
    print('h0 hipothesis is incorrect')
else:
    print('h0 hipothesis is correct')

unneeded, p = stats.mannwhitneyu(data['touch_screen'], data['dual_sim'])
print(p)  # p parameter for mannwhitneyu
if p > 0.05:
    print('h0 hipothesis is incorrect')
else:
    print('h0 hipothesis is correct')
\end{lstlisting}   

Получившиеся результаты:

Гипотеза $H_{01}$ верна - количество телефонов с поддержкой 3G отличается от кол-ва моделей с WI-Fi по двухвыборочному тесту Стьюдента

Гипотеза $H_{02}$ неверна - количество телефонов с touch screen не отличается от количества моделей с двумя сим-картами по тесту Манна-Уитни, сравнивающему средние результаты.

3. Проверить зависимость цены от объёма оперативной памяти. Гипотеза $H_0$ - Цена зависит от объема:

Для проверки данной гипотезы можем использовать критерий независимости Пирсона, точный тест Фишера, ранговый критерий Спирмена, ранговый критерий Кендалла

\begin{lstlisting}[language=Python, mathescape=true, breaklines=true]
unneeded, p = spearmanr(data['price_range'], data['ram'])
if p < 0.05:
    print('h0 is correct')
else:
    print("h0 is incorrect")
\end{lstlisting}   

Получившиеся результаты:
Применил критерии Спирмена и критерий Кендалла - гипотеза принимается (есть зависимость на статистике цены от объёма оперативной памяти)
\end{document}